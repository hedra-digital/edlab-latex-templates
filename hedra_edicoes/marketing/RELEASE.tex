\documentclass[11pt,a4paper]{article}

% Pacotes necessários
\usepackage[utf8]{inputenc}
\usepackage[portuguese]{babel}
\usepackage[left=20mm,right=20mm,top=21mm,bottom=21mm,paperwidth=210mm,paperheight=297mm]{geometry}
\usepackage{xcolor}
\usepackage{graphicx}
\usepackage{microtype}
\usepackage{setspace}
\usepackage{titlesec}
\usepackage{wrapfig}
\usepackage{calc}
\usepackage{fontspec}
\usepackage{adforn}
\usepackage{multicol}


\newcommand{\asterisc}{\begin{center}\adforn{14}\end{center}}


% Definir cor de fundo
\definecolor{almondcolor}{cmyk}{0,0.09,0.018,0.07}
\pagecolor{almondcolor}

% Configuração de fonte
\setmainfont[Ligatures=TeX,Numbers=OldStyle]{Formular}

% Alinhamento à esquerda
%\raggedright

% Configuração de espaçamento
\setstretch{1.2}

% Configuração de indentação
\setlength{\parindent}{3ex}

% Remover numeração de páginas
\pagestyle{empty}

% Configuração de seções
\titleformat{\section}
{\normalfont\large\bfseries}
{}
{0pt}
{}

% Configuração do título principal
\newcommand{\maintitle}[1]{%
\noindent\begin{minipage}{11cm}
{\LARGE\bfseries #1}\bigskip
\end{minipage}
}

% Configuração da linha fina
\newcommand{\linhafina}[1]{%
\noindent\begin{minipage}{11cm}
{\itshape #1}
\end{minipage}
}

% Comando para informações do livro na margem
\newcommand{\bookinfo}[6]{%
\begin{wrapfigure}{r}{40mm}
\includegraphics[width=50mm]{./THUMB.pdf}
\smallskip

\textbf{Título}\\ \textit{#1}\\
\textbf{Autor} #2\\
\textbf{Editora} #3\\
\textbf{ISBN}\\ #4\\
\textbf{Pág.} #5\\
\textbf{Preço} #6
\end{wrapfigure}
}

\begin{document}

% Título principal
\maintitle{Uma frase impactante\\ sobre o melhor livro do mundo}

% Linha fina em itálico
\linhafina{
Lorem ipsum dolor sit amet, consectetur adipisicing elit, sed do eiusmod
tempor incididunt ut labore et dolore magna aliqua. Ut enim ad minim veniam,
quis nostrud exercitation ullamco laboris nisi ut aliquip ex ea commodo
consequat.}
 

\vspace{1cm}

% Informações do livro na margem direita
\bookinfo{O nome do livro}{Autor da Silva}{Hedra}{978-65-89705-46-8}{120}{46}
% prefácio Tales Ab'Sáber

\noindent 
Lorem ipsum dolor sit amet, consectetur adipisicing elit, sed do eiusmod
tempor incididunt ut labore et dolore magna aliqua. Ut enim ad minim veniam,
quis nostrud exercitation ullamco laboris nisi ut aliquip ex ea commodo
consequat. Duis aute irure dolor in reprehenderit in voluptate velit esse
cillum dolore eu fugiat nulla pariatur. Excepteur sint occaecat cupidatat non
proident, sunt in culpa qui officia deserunt mollit anim id est laborum.
Lorem ipsum dolor sit amet, consectetur adipisicing elit, sed do eiusmod
tempor incididunt ut labore et dolore magna aliqua. Ut enim ad minim veniam,
quis nostrud exercitation ullamco laboris nisi ut aliquip ex ea commodo
consequat. Duis aute irure dolor in reprehenderit in voluptate velit esse
cillum dolore eu fugiat nulla pariatur. Excepteur sint occaecat cupidatat non
proident, sunt in culpa qui officia deserunt mollit anim id est laborum.


Lorem ipsum dolor sit amet, consectetur adipisicing elit, sed do eiusmod
tempor incididunt ut labore et dolore magna aliqua. Ut enim ad minim veniam,
quis nostrud exercitation ullamco laboris nisi ut aliquip ex ea commodo
consequat. Duis aute irure dolor in reprehenderit in voluptate velit esse
cillum dolore eu fugiat nulla pariatur. Excepteur sint occaecat cupidatat non
proident, sunt in culpa qui officia deserunt mollit anim id est laborum.
Lorem ipsum dolor sit amet, consectetur adipisicing elit, sed do eiusmod
tempor incididunt ut labore et dolore magna aliqua. Ut enim ad minim veniam,
quis nostrud exercitation ullamco laboris nisi ut aliquip ex ea commodo
consequat. Duis aute irure dolor in reprehenderit in voluptate velit esse
cillum dolore eu fugiat nulla pariatur. Excepteur sint occaecat cupidatat non
proident, sunt in culpa qui officia deserunt mollit anim id est laborum.
\section{Sobre o autor}

\small
\textbf{Autor a Silva} é doutor em Lorem ipsum dolor sit amet, consectetur
 adipisicing elit, sed do eiusmod tempor incididunt ut labore et dolore magna
 aliqua. Ut enim ad minim veniam, quis nostrud exercitation ullamco laboris
 nisi ut aliquip ex ea commodo consequat. Duis aute irure dolor in
 reprehenderit in voluptate velit esse cillum dolore eu fugiat nulla
 pariatur. Excepteur sint occaecat cupidatat non proident, sunt in culpa qui
 officia deserunt mollit anim id est laborum.

\pagebreak

\section{Trechos do livro}

\begin{multicols}{2}

Mas, por voltas anacrônicas da história, daquelas de arrepiar os pelos de
imberbes historiadores positivistas (e eles são muitos), o bastardo pardo do
século \textsc{xx} pode estar no século \textsc{xix}, e o senhor de engenho do \textsc{xix} pode estar
no século \textsc{xx} — e ambos, bastardo e senhor, podem estar vivíssimos no século
\textsc{xxi}. Essas duas figuras — por sinal, conceitos jurídicos — compõem o
filamento genético pátrio desde que o Brasil é Brasil. Compreendê-las,
portanto, significa compreender a formação do país; estudá-las de perto
potencializa a compreensão do ``Negro drama'' dos Racionais e as razões do
drama de Luiz Gama, duas expressões de um mesmo fenômeno ``antigo e moderno'',
quiçá mesmo ``imortal'', que tem chão no Brasil. 

O filho da ``madame nagô'' Luiza Mahin, assim como o da ``adusta Líbia rainha''
Ana Soares, conhecia na carne a dor do abandono paterno. Diferentemente,
porém, de Mano Brown, que não conheceu o pai, Gama conviveu com o seu por
nove anos. Se um foi abandonado recém-nascido, o outro o foi ainda na
infância. Ambos nunca esqueceram o que o pai fez. Ambos, ainda, refletiram
muito sobre o que significava ser ``um bastardo, mais um filho pardo, sem
pai'', no inferno intersecular do Brasil. Ambos, finalmente, falaram o mínimo
do mínimo sobre o pai, daí que o adjetivo ``canalha'', com o qual Brown definiu
o seu, e o ``manto de silêncio'' que Gama jogou sobre o nome do fidalgo,
justamente para ``poupar à sua infeliz memória uma injúria dolorosa'', são
bastante eloquentes para analisar a dialética do bastardo-pardo e do
senhor-branco.
\end{multicols}

\asterisc

\begin{multicols}{2}

Quando escreveu a carta que se tornaria sua autobiografia, em 25 de julho de
1880, Luiz Gama fez questão de sublinhar, já no preâmbulo, que puxaria de
cabeça as histórias que viriam pela frente. É como se todos aqueles
apontamentos — ``que sempre eu os trouxe de memória'' — estivessem por muito
tempo guardados a sete chaves no porão de sua casa e ele agora resolvesse
mostrá-los ao destinatário, Lúcio de Mendonça. 

O leitor da autobiografia de Gama, portanto, sente-se como o narrador de ``O
Aleph'', que visita o porão de Daneri e lá se maravilha — e se horroriza —
com o que vê. Como lembra Borges, ``Senti infinita veneração, infinita
pena'', o mesmo sentimento que toma quem lê a autobiografia de Gama. Feito o
Aleph de Daneri, porém, o de Gama também precisa de sua casa. A autobiografia
bem pode ser o cristal que projeta ``todos os pontos do universo'' de Gama,
mas tudo principia com a casa.  O sobrado de São Salvador é a chave para o
infinito particular de Gama — e é com ele que inicia sua autobiografia. Ir
até ao casarão da rua do Bângala, portanto, e descer ao seu porão significa
ver, se os olhos permitirem, os mais de mil ``lances doridos'' da ``vida
amargurada'' do filho de Luiza Mahin. 
\end{multicols}



\asterisc

\begin{multicols}{2}

Aí está a perspicácia do autor da genial narrativa autobiográfica: ele torna o
próprio caso menor para que vejamos o tráfico. Ele se exclui, e na própria
autobiografia!, para enxergarmos o ``sistema embriagado'' que permite a
``embriaguez'' do pai, a do amigo do pai, e a do parceiro novo, naquela
operação bêbada, do pai. No fundo, o pai é o de menos, assim como ele também
se torna de menos, diante da monstruosidade do holocausto do tráfico. Indicar
o nome da embarcação, o que fatalmente leva para o de seu capitão, tem,
portanto, um duplo propósito: por um lado, retraçar o seu processo individual
de escravização no fatídico 10 de novembro de 1840, e, por outro, jogar luz
no mal alumiado — e não menos seu — processo coletivo de escravização de
todos os dias e em todas as horas do Império do Brasil.   
\end{multicols}


\end{document}