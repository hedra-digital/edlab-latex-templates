%!TEX root=LIVRO.tex

% Tamanhos
% \tiny
% \scriptsize
% \footnotesize
% \small 
% \normalsize
% \large 
% \Large 
% \LARGE 
% \huge
% \Huge

% Posicionamento
% \centering 
% \raggedright
% \raggedleft
% \vfill 
% \hfill 
% \vspace{Xcm}   % Colocar * caso esteja no começo de uma página. Ex: \vspace*{...}
% \hspace{Xcm}

% Estilo de página
% \thispagestyle{<<nosso>>}
% \thispagestyle{empty}
% \thispagestyle{plain}  (só número, sem cabeço)
% https://www.overleaf.com/learn/latex/Headers_and_footers

% Compilador que permite usar fonte de sistema: xelatex, lualatex
% Compilador que não permite usar fonte de sistema: latex, pdflatex

% Definindo fontes
% \setmainfont{Times New Roman}  % Todo o texto
% \newfontfamily\avenir{Avenir}  % Contexto

\begingroup\thispagestyle{empty}\vspace*{.05\textheight} 

            {\formular
              \huge
              \noindent
              \textbf{Título}\\
              
              \vspace{-0.5cm}
              
              \noindent{}{\LARGE Título na língua indígena}}
             
                 \vspace{1cm}
              
              {\formular\Large
              \noindent{}Autor
              }

              \vfill              

              {\small
              \noindent{}Colaborador 1 (\textit{organização e
              apresentação})\\
              \noindent{}Colaborador 2 (\textit{posfácio})
              }

              \vspace{0.5cm}

              {\small\noindent{}Xª edição}

              \vfill

              \newfontfamily\timesnewroman{Times New Roman}
              {\noindent\fontsize{30}{40}\selectfont \timesnewroman hedra}

              {\selectfont\small\noindent São Paulo \quad\the\year}

\endgroup
\pagebreak
