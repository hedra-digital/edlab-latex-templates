% Tamanhos
% \tiny
% \scriptsize
% \footnotesize
% \small 
% \normalsize
% \large 
% \Large 
% \LARGE 
% \huge
% \Huge

% Posicionamento
% \centering 
% \raggedright
% \raggedleft
% \vfill 
% \hfill 
% \vspace{Xcm}   % Colocar * caso esteja no começo de uma página. Ex: \vspace*{...}
% \hspace{Xcm}

% Estilo de página
% \thispagestyle{<<nosso>>}
% \thispagestyle{empty}
% \thispagestyle{plain}  (só número, sem cabeço)
% https://www.overleaf.com/learn/latex/Headers_and_footers

% Compilador que permite usar fonte de sistema: xelatex, lualatex
% Compilador que não permite usar fonte de sistema: latex, pdflatex

% Definindo fontes
% \setmainfont{Times New Roman}  % Todo o texto
% \newfontfamily\avenir{Avenir}  % Contexto

\begingroup\thispagestyle{empty}\vspace*{-.01\textheight}\parindent=0pt 
              \formular
              \Huge 
              \textbf{Título}\baselineskip=.67\baselineskip 

              \vspace{15mm}
              
              
              \vspace{5cm}

              \newfontfamily\minion{Minion Pro}
              {\selectfont\minion\small
              Colaborador 1 (\textit{organização})}
              
              {\selectfont\minion\footnotesize
              Xª edição}
                    
              \vfill

              \newfontfamily\timesnewroman{Times New Roman}
              {\fontsize{30}{40}\selectfont \timesnewroman hedra}
              
              \medskip

              {\selectfont\minion\small
              São Paulo \quad\the\year}
\endgroup
\pagebreak

\begingroup 

\footnotesize\parindent0pt\parskip5pt\thispagestyle{empty} 
\vspace*{.1\textheight}\mbox{} \vfill
\baselineskip=.92\baselineskip
\thispagestyle{empty}

\textbf{Anarquistas na América do Sul} é uma antologia de 18 textos, escritos a partir do encontro de mesmo nome, resultado da livre associação de três núcleos de pesquisa: Nu-Sol (Núcleo de Sociabilidade Libertária da \textsc{puc--sp}), \textsc{lima} (Laboratório Insurgente de Maquinarias Anarquistas) e \textsc{lasi}n\textsc{t}ec (Laboratório de Análise em Segurança Internacional e Tecnologias de Monitoramento, da \textsc{unifesp}). Durante quatro dias de mesas e rodas de conversa virtuais, pesquisadores anarquistas de língua portuguesa e espanhola discutiram e experimentaram lutas e práticas libertárias. O livro, que contém o registro escrito de muitos desses debates, foi organizado de forma aberta, de modo a inscrever em sua estrutura o espírito livre do encontro --- documento fundamental da pesquisa libertária nesta região do planeta.

\textbf{Edson Passeti} é professor livre-docente do Departamento de
Ciências Sociais e Programa de Estudos Pós-Graduados em Ciências Sociais
da Pontifícia Universidade Católica de São Paulo \textsc{puc-sp}, onde também coordena o Nu-Sol (Núcleo de Sociabilidade Libertária). Edita o \textit{Observatório Ecopolítica}.

\textbf{Sílvio Gallo} é professor titular da Faculdade de Educação da Universidade Estadual de Campinas (Unicamp) e pesquisador bolsista de produtividade do Conselho Nacional de Desenvolvimento Científico e Tecnológico (\textsc{cnp}q). Graduado em Filosofia pela Pontifícia Universidade Católica de Campinas (\textsc{puc}--\,Campinas), realizou seus estudos de mestrado e doutorado na Faculdade de Educação da Unicamp, onde obteve também a livre-docência. Integrante do \textsc{lima}, Laboratório Insurgente de Maquinarias Anarquistas.

\textbf{Acácio Augusto} é doutor em Ciência Sociais, com enfoque em política, pela Pontifícia Universidade Católica de São Paulo
\textsc{puc--sp}, onde também é pesquisador do Nu-Sol (Núcleo de Sociabilidade Libertária). Professor no Departamento de Relações Internacionais da Universidade Federal de São Paulo \textsc{unifesp}, onde coordena o \textsc{lasi}n\textsc{t}ec (Laboratório de Análise em Segurança Interacional de Tecnologias de Monitoramento) --- e também no Programa de Pós-Graduação em Psicologia Institucional da Universidade Federal do Espírito Santo \textsc{ufes}. Atualmente coordena o curso de Relações Internacionais da \textsc{unifesp}.

\endgroup
\pagebreak